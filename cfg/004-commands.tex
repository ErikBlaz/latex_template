\newcommand{\itemtitle}[1]{\vspace{5pt}\hspace{30pt}\color{pcolor}\fontinter\normalsize{#1}\normalsize{}\vspace{5pt}\color{black}\fontnow}
\newcommand{\itemtitlebf}[1]{\vspace{5pt}\hspace{30pt}\color{pcolor}\fontinter\normalsize{\textbf{#1}}\normalsize{}\vspace{5pt}\color{black}\fontnow}
\newcommand{\itemtitleit}[1]{\vspace{5pt}\hspace{30pt}\color{pcolor}\normalsize{\textitc{#1}}\normalsize{}\vspace{5pt}\color{black}}
    \newcommand{\todo}[1]{\color{red}\textbf{\textit{TODO: }}\textit{#1}\color{black}}
\newcommand{\centerl}[1]{\begin{center}#1\end{center}}

% \renewcommand{\tablename}{Tabla} 
% \renewcommand{\listfigurename}{Lista de \figurename s}
% \renewcommand{\figurename}{Figura}
% \renewcommand{\listtablename}{Lista de \tablename s}
\renewcommand{\lstlistingname}{Código}
\renewcommand{\lstlistlistingname}{Índice de \lstlistingname s}

% \renewcommand{\appendix}{Anexo}
\newcommand{\supi}[1]{\textsuperscript{#1}}
\newcommand{\textbfc}[1]{\color{pcolor}\textbf{#1}\color{black}\xspace}
\newcommand{\textc}[1]{\color{pcolor}#1\color{black}\xspace}
\newcommand{\nosep}{\itemsep=3pt\topsep=0pt\partopsep=0pt\parskip=0pt\parsep=0pt}
% \renewcommand{\contentsname}{Lista de Contenidos}

\definecolor{light-gray}{gray}{0.95}
\newcommand{\code}[1]{\colorbox{light-gray}{\fontsize{8}{10}\fontmenlo{#1}}}

\newcommand{\listheaderfont}{\scriptsize\normalfont\scshape\color{red}}

\newcommand{\ex}{\color{pcolor}\textbf{Ejemplo:}\color{black}}
\newcommand{\textitc}[1]{\color{pcolor}\textit{#1}\color{black}\xspace}
\newcommand{\abs}[1]{\lvert#1\rvert}
\newcommand{\limg}[2]{
\noindent\begin{minipage}{0.5\textwidth}
        #1
\end{minipage}
\hspace{0.05\textwidth}
\begin{minipage}{0.4\textwidth}
        #2
\end{minipage}
}
\newcommand{\rimg}[2]{
\noindent\begin{minipage}{0.4\textwidth}
    #2
\end{minipage}
\hspace{0.05\textwidth}
\begin{minipage}{0.5\textwidth}
    #1
\end{minipage}
}

\renewcommand{\tab}{\hspace{1cm}}

\newcommand{\codeblock}[1]{
    \begin{lstlisting}
        #1
    \end{lstlisting}
}

\newcommand{\figurefont}{\fontnow\setstretch{1.20}}
\newenvironment{equationenv}[0]{}{}

\newcounter{exe_section}


\newenvironment{exercise}[1]{
\captionof{exercises}{#1}
\setcounter{exe_section}{0}
\vspace{5pt}

}{
    \setlength{\parsep}{5pt}%

}

\newenvironment{exe_section}[2]{
\stepcounter{exe_section}
\textbfc{\alph{exe_section})} \textit{#1}
\vspace{5pt}
\def\result{#2}
}{
    % \newline
    \textbfc{Resultado: }\result
    \vspace{5pt}


}

\DeclareMathOperator*{\coma}{\text{, }}
\DeclareMathOperator*{\cte}{\ cte\ }
\DeclareMathOperator*{\tq}{\text{ | }}
\DeclareMathOperator*{\diver}{divergente}
\DeclareMathOperator*{\conv}{convergente}

\newcommand{\var}[1]{\textc{$#1$}}
\newcommand{\difpart}[3]{\left(\frac{\partial #1}{\partial #2}\right)_{#3}}
\newcommand{\difpartb}[3]{\left(\frac{\partial}{\partial #2} \left(#1\right)\right)_{#3}}
\newcommand{\difpartbn}[3]{\left(\frac{\partial}{\partial #2} #1\right)_{#3}}

\def\dbar{{\mkern6mu\mathchar'26\mkern-11mu d}}
% \def\dbar{{\mathchar'26\mkern-12mu d}}

\newcommand{\df}{\color{pcolor}\textbf{Definición:\ }\color{black}}
\newcommand{\dem}{\color{pcolor}\textbf{Demostración:\ }\color{black}}
\newcommand{\prop}{\color{pcolor}\textbf{Proposición:\ }\color{black}}
\newcommand{\col}{\color{pcolor}\textbf{Colorario:\ }\color{black}}
\newcommand{\lem}{\color{pcolor}\textbf{Lema:\ }\color{black}}